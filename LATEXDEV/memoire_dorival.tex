\documentclass{report}
\usepackage{graphicx}

\begin{document}

\begin{titlepage}
    \centering
    \includegraphics[width=0.3\textwidth]{logo.jpg}\par\vspace{1cm}
    {\scshape\Large Université Des Antilles \par}
    \vspace{1cm}
    {\scshape\Large Master Mathématiques et Applications \par}
    \vspace{3cm}
    {\huge\bfseries Titre de votre mémoire \par}
    \vspace{3cm}
    {\Large\itshape Votre Nom \par}
    \vfill
    Supervisé par\par
    Nom de l'Encadreur

    \vfill

    % Bottom of the page
    {\large \today\par}
\end{titlepage}

\chapter*{Remerciements}
\addcontentsline{toc}{chapter}{Remerciements}

Insérez remerciements ici.

\tableofcontents
\newpage

\chapter*{Introduction}
\addcontentsline{toc}{chapter}{Introduction}

La Martinique, île des Caraïbes réputée pour sa biodiversité et son agriculture, est confrontée depuis plusieurs décennies à un problème environnemental majeur : la contamination des sols agricoles par le chlordécone. Utilisé massivement dans les plantations de bananes entre les années 1970 et 1993, le chlordécone, pesticide organochloré persistant, persiste dans l'environnement et représente une menace pour la santé publique et la sécurité alimentaire.

La dégradation du chlordécone dans les sols agricoles est un processus complexe influencé par une multitude de facteurs environnementaux. Comprendre ces facteurs et leur interaction est essentiel pour élaborer des stratégies efficaces de gestion et de décontamination des terres agricoles martiniquaises.

La présente étude se propose d'explorer en profondeur cette problématique en se concentrant sur les principaux facteurs environnementaux qui influencent la dégradation du chlordécone dans les sols agricoles de la Martinique sur une période de 15 ans. Plus précisément, nous chercherons à répondre à la question suivante : quels sont les facteurs environnementaux qui interagissent pour influencer la dégradation du chlordécone, et comment ces interactions affectent-elles ce processus au fil du temps ?

Pour répondre à cette question, nous adopterons une approche multidisciplinaire, combinant des analyses statistiques avancées, des techniques de modélisation environnementale et une exploration approfondie des données longitudinales sur les niveaux de chlordécone et les variables environnementales pertinentes. Cette approche nous permettra d'identifier les tendances principales, de caractériser les facteurs environnementaux clés et d'évaluer leur impact sur la dégradation du chlordécone dans les sols agricoles martiniquaises.

En fin de compte, cette recherche vise à enrichir notre compréhension des processus de dégradation du chlordécone dans un contexte insulaire et tropical, et à fournir des connaissances fondamentales pour orienter les politiques de gestion environnementale et les pratiques agricoles durables dans la région de la Martinique.

\chapter*{État des connaissances sur la dégradation du chlordécone dans les sols agricoles martiniquais.}

\section{Effets du chlordécone sur l'environnement et la santé humaine}

Le chlordécone, un pesticide organochloré largement utilisé dans les plantations de bananes en Martinique jusqu'au début des années 1990, est connu pour ses effets néfastes sur l'environnement et la santé humaine. Des études ont montré sa persistance dans les sols agricoles, son accumulation dans la chaîne alimentaire et son association avec des risques pour la santé, notamment des effets neurotoxiques et cancérigènes chez l'homme.

\section{Processus de dégradation du chlordécone}


La dégradation du chlordécone dans les sols agricoles est un processus complexe influencé par divers facteurs environnementaux. Les études précédentes ont identifié plusieurs voies de dégradation, notamment la biodégradation par des microorganismes du sol, l'adsorption sur les particules du sol et la dégradation photochimique sous l'effet de la lumière solaire. Cependant, la compréhension de ces processus reste incomplète, en particulier en ce qui concerne leur cinétique et leurs mécanismes exacts.

\section{Facteurs environnementaux influençant la dégradation du chlordécone}


Plusieurs études ont examiné l'impact des facteurs environnementaux sur la dégradation du chlordécone dans les sols agricoles. Ces facteurs comprennent le type de sol, l'exposition solaire, la température, l'humidité du sol, le pH du sol et la présence de microorganismes dégradateurs. Cependant, les résultats de ces études sont parfois contradictoires, ce qui souligne la nécessité d'une analyse approfondie et intégrée des facteurs environnementaux pour comprendre pleinement leur rôle dans le processus de dégradation du chlordécone.

\section{Lacunes}


Malgré les progrès réalisés dans la compréhension de la dégradation du chlordécone, plusieurs lacunes subsistent. Ces lacunes incluent le manque de données longitudinales à long terme sur la dégradation du chlordécone dans les sols agricoles de la Martinique, ainsi que l'absence d'études intégrant de manière exhaustive les différents facteurs environnementaux influençant ce processus.
\\

Dans cette section,on met en évidence l'importance cruciale de comprendre les processus de dégradation du chlordécone dans les sols agricoles martiniquaises, ainsi que les facteurs environnementaux qui les influencent. Cette compréhension est essentielle pour élaborer des stratégies efficaces de gestion et de décontamination des terres agricoles contaminées par le chlordécone dans la région de la Martinique.



\section{Méthodologie}

La méthodologie adoptée dans cette étude est cruciale pour répondre à la problématique posée et atteindre les objectifs fixés. Cette section détaille les données utilisées, les méthodes analytiques appliquées et les étapes suivies pour mener à bien l'analyse des facteurs environnementaux influençant la dégradation du chlordécone dans les sols agricoles de la Martinique sur une période de 15 ans.

\subsection{Données utilisées}

Les données utilisées dans cette étude comprennent des séries temporelles longitudinales détaillant les niveaux de chlordécone dans les sols agricoles de la Martinique sur une période de 15 ans, de 2004 à 2019. Ces données ont été collectées à partir de différentes sources, y compris les relevés sur le terrain, les bases de données gouvernementales et les études de recherche précédentes. En outre, des données environnementales telles que la pluviométrie, l'exposition solaire, la rugosité du terrain et d'autres variables pertinentes ont été collectées pour chaque site d'échantillonnage.

**3.2. Méthodes analytiques**

Pour analyser les données longitudinales sur les niveaux de chlordécone et les variables environnementales, plusieurs méthodes analytiques ont été utilisées. Tout d'abord, une analyse statistique descriptive a été réalisée pour explorer les tendances temporelles et spatiales des concentrations de chlordécone dans les sols agricoles de la Martinique. Ensuite, des modèles statistiques adaptés aux données longitudinales ont été appliqués pour évaluer l'impact des facteurs environnementaux sur la dégradation du chlordécone. Ces modèles ont permis de prendre en compte la structure temporelle des données et les interactions entre les différentes variables.

En outre, des techniques de clustering et de classification dynamique ont été employées pour regrouper les parcelles en fonction de leur profil de décontamination du chlordécone. Ces techniques ont permis d'identifier des groupes de terres présentant des caractéristiques similaires en termes de vitesse et de pattern de dégradation du chlordécone, ce qui a facilité la caractérisation des différents profils de décontamination.

**3.3. Stratégies d'évaluation des résultats**

Pour évaluer les résultats obtenus, plusieurs stratégies ont été mises en œuvre. Tout d'abord, des analyses de sensibilité ont été réalisées pour évaluer la robustesse des modèles statistiques aux variations des paramètres et des conditions initiales. Ensuite, des méthodes de validation croisée ont été utilisées pour évaluer la capacité des modèles à prédire les concentrations de chlordécone dans les sols agricoles à partir des données environnementales. Enfin, les groupes de parcelles identifiés par les techniques de clustering et de classification dynamique ont été examinés en détail pour comprendre les facteurs environnementaux ou les caractéristiques spécifiques associés à chaque groupe.

En conclusion, la méthodologie adoptée dans cette étude a permis de mener une analyse approfondie des facteurs environnementaux influençant la dégradation du chlordécone dans les sols agricoles de la Martinique. Cette approche multidisciplinaire, combinant des méthodes statistiques avancées et des techniques de clustering, a permis de caractériser les différents profils de décontamination du chlordécone et d'identifier les principaux facteurs environnementaux associés à ce processus.
\end{document}
