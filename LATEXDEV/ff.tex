






\documentclass[11pt,twoside,a4paper]{article}
%=============== En−Tete ===============
%−−− Insertion de paquetages (optionnel) −−−
\usepackage[french]{babel} % pour dire que le texte est en fran¸cais

\usepackage{a4} % pour la taille
\usepackage[T1]{fontenc} % pour les font postscript
\usepackage[cyr]{aeguill} % Police vectorielle TrueType, guillemets fran¸cais
\usepackage{epsfig} % pour g´erer les images
\usepackage{amsmath, amsthm} % tr`es bon mode math´ematique
\usepackage{amsfonts,amssymb}% permet la definition des ensembles
\usepackage{float} % pour le placement des figure
\usepackage{url} % pour une gestion efficace des url
\bibliographystyle{plain} % Style bibliographique
%=============== Corps ===============
\begin{document}

\tableofcontents % ´ecrit la table des mati`eres

% Page de couverture
\begin{titlepage}
    \centering
    {\LARGE\bfseries Université des Antilles\par}
    \vspace{2cm}
    {\scshape\Large Dorival Pierre Chrislin\par}
    \vspace{2cm}
    {\large\itshape\today\par}
\end{titlepage}

% Page de formules mathématiques avec démonstration du théorème de Cauchy
\section*{Résoluton Exercices 1}

On considère un modèle à générations imbriquées où les ménages ne vivent
que deux périodes. Au cours de leur période d'activité, ils perçoivent un
salaire $w_t$ qu'ils affectent à la consommation $c_t$ courante et à l'épargne $s_t$.
Au cours de leur période de retraite, ils consacrent leur épargne à la consom-
mation différée $d_{t+1}$. Leurs préférences intertemporelles sont représentées par
la fonction d'utilité :


$u_t(c_t; d_{t+1})=\frac{1}{2}\log{c_t} + \frac{1}{2}\log{d_{t+1}}$


Dans cette économie, les impôts sont nuls $(x = 0)$. Les individus nés en $t$
sont en nombre $L_t$. Le taux de croissance du nombre de travailleurs d'une
génération à l'autre est noté $n = 1$:

$a)$ Exprimons les consommations des jeunes et des vieux en fonction du saliare
$wt$:

Fonction d'utilité compte tenu de la contrainte budgétaire intertemporelle:

$ \left\{\begin{array}{rl}
 Max \ u_t(c_t; d_{t+1})=  \frac{1}{2}\log{c_t} + \frac{1}{2}\log{d_{t+1}} \\
 
w_t  =  c_t + \frac{d_{t+1}}{1+r_{t+1}}\end{array}\right. $  \\

Considérons la fonction Lagrangienne :

$ L(c_t,d_{t+1},\lambda_t)= \frac{1}{2}\log{c_t} + \frac{1}{2}\log{d_{t+1}} +\lambda_t(w_t-c_t - \frac{d_{t+1}}{1+r_{t+1}})  $  \\

Faisons la dérivée de la fonction Lagrangienne par rapport $c_t :  $ \\

$\\
\frac{\partial L(c_t,d_{t+1},\lambda_t)}{\partial c_t} = \frac{1}{2c_t} - \lambda_t 
$ \\
Posons  $ \frac{\partial L(c_t,d_{t+1},\lambda_t)}{\partial c_t} = 0  $ \\



\begin{equation}\label{eq:test1}
  \Rightarrow \lambda_t=\frac{1}{2c_t}
\end{equation} 


Faisons la dérivée de la fonction Lagrangienne par rapport $d_{t+1} : \\ $
$\\
\frac{\partial L(c_t,d_{t+1},\lambda_t)}{\partial d_{t+1}} = \frac{1}{2 d_{t+1}} - \frac{\lambda_t}{1+r_{t+1}} 
$\\

Posons  $ \frac{\partial L(c_t,d_{t+1},\lambda_t)}{\partial d_{t+1}} = 0 $ \\

  \begin{equation}\label{eq:test2}
  \Rightarrow \lambda_t=\frac{1+r_{t+1}}{2 d_{t+1}}
\end{equation}  

Faisons la dérivée de la fonction Lagrangienne par rapport $\lambda_t :  $ \\
$\\
\frac{\partial L(c_t,d_{t+1},\lambda_t)}{\partial \lambda_t} = w_t-c_t - \frac{d_{t+1}}{1+r_{t+1}} 
$ \\

Posons  $ \frac{\partial L(c_t,d_{t+1},\lambda_t)}{\partial \lambda_t} = 0 $ \\

 \begin{equation}\label{eq:test3}
  \Rightarrow w_t=c_t + \frac{d_{t+1}}{1+r_{t+1}}
\end{equation}
    

(\ref{eq:test1}) et (\ref{eq:test2}) nous donnent : \\
$\\
\frac{1}{c_t} = \frac{1+r_{t+1}}{d_{t+1}}\\
$\\
$ \Rightarrow c_t= \frac{d_{t+1}}{1+r_{t+1}}$  et  $ d_{t+1}= c_t ({1+r_{t+1}})$\\

Dans (\ref{eq:test1})\: $ w_t=c_t + \frac{d_{t+1}}{1+r_{t+1}}$ \\

D'o\'{u} \:
 \fbox {$c_t = \frac{1}{2} w_t $} \: et \: \fbox {$d_{t+1} = \frac{1}{2} w_t(1+r_{t+1}) $}\\
 
 
 $b)$  Déterminer l'épargne en fonction de $w_t$.\\
 
 On sait que: $ s_t = \frac{d_{t+1}}{1+r_{t+1}} $\\
 
 $ \Rightarrow$  \fbox {$s_t = \frac{1}{2} w_t $} \\
 
 La technologie est représentée par la fonction de production :\\
 
 $Y_t=10K_t^{\frac{1}{2}}L_t^{\frac{1}{2}}$\\
 
 avec $K_t$ est le stock de capital et $L_t$ l'emploi.\\
 
 $c)$ Supposant les marchés concurrentiels, exprimons le salaire et les taux d'intér\^{e}t
en fonction du stock de capital par t\^{e}te.\\

$ max\ \pi_t= Y_t -w_tL_t-K_t r_t $\\

 $ max\ \pi_t=10K_t^{\frac{1}{2}}L_t^{\frac{1}{2}}-w_tL_t-K_t r_t $\\
 
 $
  \frac{\partial (max\ \pi_t)}{\partial K_t}=5K_t^{\frac{-1}{2}}L_t^{\frac{1}{2}}-r_t $\\
 
$ \Rightarrow r_t= 5(\frac{K_t}{L_t})^{\frac{-1}{2}}=5k_t^{\frac{-1}{2}}$ \: avec $k_t$ stock capital par t\^{e}te.\\


 $ \Rightarrow$  \fbox {$r_t = 5k_t^{\frac{-1}{2}} $} \\
 \\Calcul maintenant $\frac{\partial (max\ \pi_t)}{\partial L_t}$\\
 
 $\frac{\partial (max\ \pi_t)}{\partial L_t}=5K_t^{\frac{1}{2}}L_t^{\frac{-1}{2}}-w_t$\\
 
 $ \Rightarrow r_t= 5(\frac{K_t}{L_t})^{\frac{1}{2}}=5k_t^{\frac{1}{2}}$ \: avec $k_t$ stock capital par t\^{e}te.\\

$ \Rightarrow$  \fbox {$w_t = 5k_t^{\frac{1}{2}} $} \\

$d)$ Déterminons l'évolution du stock de capital par tête.\\
$K_{t+1}=L_t s_t$ \: or \: $L_t=\frac{L_{t+1}}{N}$\\
$\Rightarrow  K_{t+1}=L_{t+1}\frac{s_t}{2}$\\
$\Rightarrow \frac{K_{t+1}}{L_{t+1}}=k_{t+1}=\frac{1}{4}w_t$ \: car $s_t = \frac{1}{2} w_t $\\
$ \Rightarrow$  \fbox {$k_{t+1} =\frac{5}{4}k_t^{\frac{1}{2}} $} \: $w_t = 5k_t^{\frac{1}{2}} $\\

$e)$ Déterminons l'expression du capital par tête à l'équilibre. Cet équilibre
est-il stable:\\

A l'équilibre: \: $k_{t+1}=k_t=k^*=\frac{5}{4}k^{*\frac{1}{2}}$\\

$ \Rightarrow$  \fbox {$k^*=\frac{25}{16}$} \\

$\log{(k_{t+1})}=\log{(\frac{5}{4}k_t^{\frac{1}{2}})}+\log{(\frac{5}{4})}+\frac{1}{2}\log{(k_t)}$\\

$
\Rightarrow \log{(k_{t+1})} -\log{(\frac{5}{4})}=\frac{1}{2}\log{(k_t)}$\\

$
\Rightarrow \log{(\widehat{k_{t+1}})}=\frac{1}{2}\log{(\widehat{k_{t}})}$\\

$
 \alpha <\frac{1}{2}<1$\\
 L'équilibre est stable\\
 
 $ f)$ La valeur du stock de capital de la règle d'or.\\
 $Y_t=10K_t^{\frac{1}{2}}L_t^{\frac{1}{2}}$\\
 $Y_t=L_tc_t+L_{t+1}K_{t+1}+L_{t-1}d_{t+1}$\\
 $\frac{Y_t}{L_t}=c_t+(1+n)K_{t+1}+\frac{d_{t+1}}{1+n}$\: car $L_{t+1}=(1+N)L_t$\\
$ \frac{Y_t}{L_t}=10(\frac{K_t}{L_t})^{\frac{1}{2}}=10k_t^{\frac{1}{2}}$\\

$ 10k_t^{\frac{1}{2}}=c_t+(1+n)K_{t+1}+\frac{d_{t+1}}{1+n}$\\
 
 $ c_t=10k_t^{\frac{1}{2}}-2K_{t+1}-\frac{d_{t+1}}{2}$\\
 
 $\frac{\partial c_t}{K_t}=5k_t^\frac{-1}{2}-2$\\
 
  Posons $\frac{\partial c_t}{K_t}=0$\\
  $k_t^\frac{-1}{2}=\frac{2}{5}$\\
  $k_t=\frac{25}{4}=k_{or}$\\
  Dans ce cas l'économie est en régime sous-accumulation.\\
  
  \section*{Résoluton Exercices 1}
  
  Considérons un modèle à générations imbriquées où les ménages ne vivent que deux périodes. Durant la première période, ils perçoivent un salaire $w_t$; qu'ils affectent à la  consommation courante $c_t$; et à l'épargne $s_t$; et au versement des cotisations de retraite, $\sigma w_t$: $\sigma$; est le taux de cotisation à la retraite par répartition. Devenus vieux à la deuxième période, ils consacrent leur épargne et leur pension de retraite $\overline{\theta}\phi w_t$; à la consommation différée $d_{t+1}$.
$\overline{\theta}$ ; est l'espérance de vie du retraité tandis que ; est le taux de remplacement
des retraites. Les préférences des ménages sont représentées par la fonction d'utilité :\\
$u_t(c_t; d_{t+1})=(1-s)\log{c_t} + s\log{d_{t+1}}$ \: $0<s<1$
\\
Notons $N$, l'effctif (constant) de la population active.\\

$a)$ Déterminons la contrainte budgétaire intertemporelle des ménages.\\

Fonction d'utilité compte tenu de la contrainte budgétaire intertemporelle:\\

$ \left\{\begin{array}{rl}
 Max \ u_t(c_t; d_{t+1})=  (1-s)\log{c_t} + s\log{d_{t+1}} ; 0<s<1 \\
 
w_t  =  c_t + s_t +  \sigma w_t \end{array}\right. $  \\

On sait que :\\

$d_{t+1}=(1+r_{t+1})s_t+\overline{\theta}\phi w_t$\\

  $
\Rightarrow s_t=\frac{d_{t+1}-\overline{\theta}\phi w_t}{(1+r_{t+1})}$\\

%D'o\'{u}:\\
%$\Rightarrow w_t  -\sigma w_t =  c_t + \frac{d_{t+1}-\overline{\theta}\phi w_t}{(1+r_{t+1})}$\\ 
%$\Rightarrow w_t = \frac{c_t + \frac{d_{t+1}-\overline{\theta}\phi w_t}{(1+r_{t+1})}}{(1-\sigma)} $\\
%D'o\'{u}, la fonction d'utilité compte tenu de la contrainte budgétaire intertemporelle:\\
%$ \left\{\begin{array}{rl}
% Max \ u_t(c_t; d_{t+1})=  (1-s)\log{c_t} + s\log{d_{t+1}} ; 0<s<1 \\
 
%\Rightarrow w_t = \frac{c_t(1+r_{t+1}) + d_{t+1}-\overline{\theta}\phi w_t}{(1-\sigma)((1+r_{t+1}))} \end{array}\right. $  \\
 
 b) Exprimeons les consommations et l'épargne en fonction du salaire et du taux
d'intér\^{e}t.\\
La technologie des firmes individuelles est représentée par la fonction de production suivante :\\
$Y_{it}=A_tK^{\alpha}_{it}L^{1-\alpha}_{it}$\\

avec \: $A_t=BK^{1-\alpha}_t$ \: $0<\alpha<1$ \: $ i=\{ 1,2,3,...,M \}$\\

$Y_{it}, K_{it},L_{it},$ ; reprÈsentent respectivement, la production, le stock de capital et l'emploi de la firme $i$. $A_t$ est l'efficacité du travail tandis que $K_t$ est assimilé au stock de capital total.\\



Considérons la fonction Lagrangienne :

$ L(c_t,d_{t+1},\lambda_t)= (1-s)\log{c_t} + s\log{d_{t+1}} +\lambda_t(w_t-c_t - \sigma w_t-\frac{d_{t+1}-\overline{\theta}\phi w_t}{(1+r_{t+1})})  $  \\

Faisons la dérivée de la fonction Lagrangienne par rapport $c_t :  $ \\

$\\
\frac{\partial L(c_t,d_{t+1},\lambda_t)}{\partial c_t} = \frac{1-s}{c_t} - \lambda_t 
$ \\
Posons  $ \frac{\partial L(c_t,d_{t+1},\lambda_t)}{\partial c_t} = 0  $ \\



\begin{equation}\label{eq:test4}
  \Rightarrow \lambda_t=\frac{1-s}{c_t}
\end{equation} 


Faisons la dérivée de la fonction Lagrangienne par rapport $d_{t+1} : \\ $
$\\
\frac{\partial L(c_t,d_{t+1},\lambda_t)}{\partial d_{t+1}} = \frac{s}{ d_{t+1}} - \frac{\lambda_t}{1+r_{t+1}} 
$\\

Posons  $ \frac{\partial L(c_t,d_{t+1},\lambda_t)}{\partial d_{t+1}} = 0 $ \\



  \begin{equation}\label{eq:test5}
  \Rightarrow \lambda_t=\frac{(1+r_{t+1})s}{ d_{t+1}}
\end{equation}  

Faisons la dérivée de la fonction Lagrangienne par rapport $\lambda_t :  $ \\
$\\
\frac{\partial L(c_t,d_{t+1},\lambda_t)}{\partial \lambda_t} = w_t-c_t - \frac{d_{t+1}}{1+r_{t+1}} 
$ \\ \\
Posons  $ \frac{\partial L(c_t,d_{t+1},\lambda_t)}{\partial \lambda_t} = 0 $ \\

$\Rightarrow w_t-c_t - \sigma w_t-\frac{d_{t+1}-\overline{\theta}\phi w_t}{(1+r_{t+1})}=0$\\

 \begin{equation}\label{eq:test6}
  \Rightarrow w_t = \frac{c_t(1+r_{t+1}) + d_{t+1}-\overline{\theta}\phi w_t}{(1-\sigma)(1+r_{t+1})}
\end{equation}
    

(\ref{eq:test4}) et (\ref{eq:test5}) nous donnent : \\

$\\
\frac{1-s}{c_t} = \frac{(1+r_{t+1})s}{ d_{t+1}}\\
$\\
$ \Rightarrow c_t= \frac{(1-s)d_{t+1}}{(1+r_{t+1})s}$  \: et \: $ d_{t+1}=c_t \frac{ s{(1+r_{t+1})}}{1-s}$\\ \\
Dans (\ref{eq:test6})\: $ w_t = \frac{c_t(1+r_{t+1}) + d_{t+1}-\overline{\theta}\phi w_t}{(1-\sigma)((1+r_{t+1}))}$\\ \\
Rempla\c{c}ons $d_{t+1}$ dans $w_t$, \\ \\
$c_t(1+r_{t+1})+\frac{c_t s{(1+r_{t+1})}}{1-s}-\overline{\theta}\phi w_t= w_t(1_\sigma)(1+r_{t+1})$ \\ \\
D'o\'{u} \:
 \fbox {$c_t =  w_t(1-s)(\frac{\overline{\theta}\phi}{1+r_{t+1}} +1- \sigma )$} \\ \\ 
 Portons $c_t$ dans $d_{t+1}$, \\ \\
 $ d_{t+1}= w_t(1-s)(\frac{\overline{\theta}\phi}{1+r_{t+1}} +1- \sigma )\frac{ s{(1+r_{t+1})}}{1-s}$ \\ \\
\fbox {$d_{t+1} = s w_t\left(\overline{\theta}\phi +(1- \sigma)(1+r_{t+1}) \right)$}\\ \\
 On sait que :  \\ \\
 $s_t=\frac{d_{t+1}-\overline{\theta}\phi w_t}{(1+r_{t+1})}$\\ \\ 
 Rempla\c{c}ons $d_{t+1}$ dans $w_t$, \\ \\
  $s_t=\frac{ w_t\left(\overline{\theta}\phi +(1- \sigma)(1+r_{t+1}) \right)-\overline{\theta}\phi w_t}{(1+r_{t+1})}$\\ \\






$c)$ Montrer que dans une économie concurrentielle, le salaire et le taux díintÈrÍt sont donnés par :





\end{document}






