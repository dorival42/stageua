






\documentclass[11pt,twoside,a4paper]{article}
%=============== En−Tete ===============
%−−− Insertion de paquetages (optionnel) −−−
\usepackage[french]{babel} % pour dire que le texte est en fran¸cais
\usepackage{a4} % pour la taille
\usepackage[T1]{fontenc} % pour les font postscript
\usepackage[cyr]{aeguill} % Police vectorielle TrueType, guillemets fran¸cais
\usepackage{epsfig} % pour g´erer les images
\usepackage{amsmath, amsthm} % tr`es bon mode math´ematique
\usepackage{amsfonts,amssymb}% permet la definition des ensembles
\usepackage{float} % pour le placement des figure
\usepackage{url} % pour une gestion efficace des url
\bibliographystyle{plain} % Style bibliographique
%=============== Corps ===============
\begin{document}

\tableofcontents % ´ecrit la table des mati`eres

% Page de couverture
\begin{titlepage}
    \centering
    {\LARGE\bfseries Université des Antilles\par}
    \vspace{2cm}
    {\scshape\Large Dorival Pierre Chrislin\par}
    \vspace{2cm}
    {\large\itshape\today\par}
\end{titlepage}

% Page de formules mathématiques avec démonstration du théorème de Cauchy
\section*{Résoluton Exercices 1}

On considère un modèle à générations imbriquées où les ménages ne vivent
que deux périodes. Au cours de leur période d'activité, ils perçoivent un
salaire $w_t$ qu'ils affectent à la consommation $c_t$ courante et à l'épargne $s_t$.
Au cours de leur période de retraite, ils consacrent leur épargne à la consom-
mation différée $d_{t+1}$. Leurs préférences intertemporelles sont représentées par
la fonction d'utilité :


$u_t(c_t; d_{t+1})=\frac{1}{2}\log{c_t} + \frac{1}{2}\log{d_{t+1}}$


Dans cette économie, les impôts sont nuls $(x = 0)$. Les individus nés en $t$
sont en nombre $L_t$. Le taux de croissance du nombre de travailleurs d'une
génération à l'autre est noté $n = 1$:

$a)$ Exprimons les consommations des jeunes et des vieux en fonction du saliare
$wt$:

\end{document}






