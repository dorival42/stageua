\documentclass{article}
\usepackage{amsmath} % Pour les formules mathématiques

\begin{document}

% Page de couverture
\begin{titlepage}
    \centering
    {\LARGE\bfseries Université des Antilles\par}
    \vspace{2cm}
    {\scshape\Large Dorival Pierre Chrislin\par}
    \vspace{2cm}
    {\large\itshape\today\par}
\end{titlepage}

% Page de formules mathématiques avec démonstration du théorème de Cauchy
\section*{Théorème de Cauchy}

Soit \( f(x) \) une fonction continue sur l'intervalle fermé \( [a,b] \) et dérivable sur l'intervalle ouvert \( (a,b) \). Alors, il existe un réel \( c \) dans l'intervalle \( (a,b) \) tel que :

\[
f(b) - f(a) = f'(c)(b-a)
\]

\begin{proof}
Par le théorèm\documentclass{article}
\usepackage{amsmath} % Pour les formules mathématiques

\begin{document}

% Page de couverture
\begin{titlepage}
    \centering
    {\LARGE\bfseries Université des Antilles\par}
    \vspace{2cm}
    {\scshape\Large Dorival Pierre Chrislin\par}
    \vspace{2cm}
    {\large\itshape\today\par}
\end{titlepage}

% Page de formules mathématiques avec démonstration du théorème de Cauchy
\section*{Théorème de Cauchy}

Soit \( f(x) \) une fonction continue sur l'intervalle fermé \( [a,b] \) et dérivable sur l'intervalle ouvert \( (a,b) \). Alors, il existe un réel \( c \) dans l'intervalle \( (a,b) \) tel que :

\[
f(b) - f(a) = f'(c)(b-a)
\]

\begin{proof}
Par le théorème des accroissements finis, il existe un \( c \) dans \( (a,b) \) tel que :

\[
f'(c) = \frac{f(b) - f(a)}{b-a}
\]

En multipliant des deux côtés par \( b-a \), on obtient le résultat souhaité :

\[
f(b) - f(a) = f'(c)(b-a)
\]
\end{proof}

\end{document}
e des accroissements finis, il existe un \( c \) dans \( (a,b) \) tel que :

\[
f'(c) = \frac{f(b) - f(a)}{b-a}
\]

En multipliant des deux côtés par \( b-a \), on obtient le résultat souhaité :

\[
f(b) - f(a) = f'(c)(b-a)
\]
\end{proof}

\end{document}
