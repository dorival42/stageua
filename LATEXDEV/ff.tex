






\documentclass[11pt,twoside,a4paper]{article}
%=============== En−Tete ===============
%−−− Insertion de paquetages (optionnel) −−−
\usepackage[french]{babel} % pour dire que le texte est en fran¸cais

\usepackage{a4} % pour la taille
\usepackage[T1]{fontenc} % pour les font postscript
\usepackage[cyr]{aeguill} % Police vectorielle TrueType, guillemets fran¸cais
\usepackage{epsfig} % pour g´erer les images
\usepackage{amsmath, amsthm} % tr`es bon mode math´ematique
\usepackage{amsfonts,amssymb}% permet la definition des ensembles
\usepackage{float} % pour le placement des figure
\usepackage{url} % pour une gestion efficace des url
\bibliographystyle{plain} % Style bibliographique
%=============== Corps ===============
\begin{document}

\tableofcontents % ´ecrit la table des mati`eres

% Page de couverture
\begin{titlepage}
    \centering
    {\LARGE\bfseries Université des Antilles\par}
    \vspace{2cm}
    {\scshape\Large Dorival Pierre Chrislin\par}
    \vspace{2cm}
    {\large\itshape\today\par}
\end{titlepage}

% Page de formules mathématiques avec démonstration du théorème de Cauchy
\section*{Résoluton Exercices 1}

On considère un modèle à générations imbriquées où les ménages ne vivent
que deux périodes. Au cours de leur période d'activité, ils perçoivent un
salaire $w_t$ qu'ils affectent à la consommation $c_t$ courante et à l'épargne $s_t$.
Au cours de leur période de retraite, ils consacrent leur épargne à la consom-
mation différée $d_{t+1}$. Leurs préférences intertemporelles sont représentées par
la fonction d'utilité :


$u_t(c_t; d_{t+1})=\frac{1}{2}\log{c_t} + \frac{1}{2}\log{d_{t+1}}$


Dans cette économie, les impôts sont nuls $(x = 0)$. Les individus nés en $t$
sont en nombre $L_t$. Le taux de croissance du nombre de travailleurs d'une
génération à l'autre est noté $n = 1$:

$a)$ Exprimons les consommations des jeunes et des vieux en fonction du saliare
$wt$:

Fonction d'utilité compte tenu de la contrainte budgétaire intertemporelle:

$ \left\{\begin{array}{rl}
 Max \ u_t(c_t; d_{t+1})=  \frac{1}{2}\log{c_t} + \frac{1}{2}\log{d_{t+1}} \\
 
w_t  =  c_t + \frac{d_{t+1}}{1+r_{t+1}}\end{array}\right. $  \\

Considérons la fonction Lagrangienne :

$ L(c_t,d_{t+1},\lambda_t)= \frac{1}{2}\log{c_t} + \frac{1}{2}\log{d_{t+1}} +\lambda_t(w_t-c_t - \frac{d_{t+1}}{1+r_{t+1}})  $  \\

Faisons la dérivée de la fonction Lagrangienne par rapport $c_t :  $ \\

$\\
\frac{\partial L(c_t,d_{t+1},\lambda_t)}{\partial c_t} = \frac{1}{2c_t} - \lambda_t 
$ \\
Posons  $ \frac{\partial L(c_t,d_{t+1},\lambda_t)}{\partial c_t} = 0  $ \\



\begin{equation}\label{eq:test1}
  \Rightarrow \lambda_t=\frac{1}{2c_t}
\end{equation} 


Faisons la dérivée de la fonction Lagrangienne par rapport $d_{t+1} : \\ $
$\\
\frac{\partial L(c_t,d_{t+1},\lambda_t)}{\partial d_{t+1}} = \frac{1}{2 d_{t+1}} - \frac{\lambda_t}{1+r_{t+1}} 
$\\

Posons  $ \frac{\partial L(c_t,d_{t+1},\lambda_t)}{\partial d_{t+1}} = 0 $ \\

  \begin{equation}\label{eq:test2}
  \Rightarrow \lambda_t=\frac{1+r_{t+1}}{2 d_{t+1}}
\end{equation}  

Faisons la dérivée de la fonction Lagrangienne par rapport $\lambda_t :  $ \\
$\\
\frac{\partial L(c_t,d_{t+1},\lambda_t)}{\partial \lambda_t} = w_t-c_t - \frac{d_{t+1}}{1+r_{t+1}} 
$ \\

Posons  $ \frac{\partial L(c_t,d_{t+1},\lambda_t)}{\partial \lambda_t} = 0 $ \\

 \begin{equation}\label{eq:test3}
  \Rightarrow w_t=c_t + \frac{d_{t+1}}{1+r_{t+1}}
\end{equation}
    

(\ref{eq:test1}) et (\ref{eq:test2}) nous donnent : \\
$\\
\frac{1}{c_t} = \frac{1+r_{t+1}}{d_{t+1}}\\
$\\
$ \Rightarrow c_t= \frac{d_{t+1}}{1+r_{t+1}}$  et  $ d_{t+1}= c_t ({1+r_{t+1}})$\\

Dans (\ref{eq:test1})\: $ w_t=c_t + \frac{d_{t+1}}{1+r_{t+1}}$ \\

D'o\'{u} \:
 \fbox {$c_t = \frac{1}{2} w_t $} \: et \: \fbox {$d_{t+1} = \frac{1}{2} w_t(1+r_{t+1}) $}\\
 
 
 $b)$  Déterminer l'épargne en fonction de $w_t$.\\
 
 On sait que: $ s_t = \frac{d_{t+1}}{1+r_{t+1}} $\\
 
 $ \Rightarrow$  \fbox {$s_t = \frac{1}{2} w_t $} \\
 
 La technologie est représentée par la fonction de production :\\
 
 $Y_t=10K_t^{\frac{1}{2}}L_t^{\frac{1}{2}}$\\
 
 avec $K_t$ est le stock de capital et $L_t$ l'emploi.\\
 
 $c)$ Supposant les marchés concurrentiels, exprimons le salaire et les taux d'intér\^{e}t
en fonction du stock de capital par t\^{e}te.\\



 
 
 
 





\end{document}






